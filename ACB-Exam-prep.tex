\documentclass[10pt]{cls/labreport}
%------------------------------------------------------------------------------------------------
% Use the documentclass option 'lineno' to view line numbers
%  COMMANDS:
% \articletype{styles} ==>>  will load  "geos, gs, iba, inv, mp" style files
% \inputfile{table,figures,...} in document ==> to load tables and figures from helper files ..
% \fullref{tab:table1 or fig:figure1} etc   ==>  to reference tables, figures, etc
% \cite(alt(p))[p.32]{Tewabe2021} ==>  (Jones et al., 1990, P32)
% \table1{\textwidth}{Caption}              ==>  insert tables
% \figure{0.34}{Caption}                    ==>  insert figures
% \twocolstart                              ==>  switch to two column mode 
% \twocolstop                               ==>  switch to one column mode 
%-------------------------------------------------------------------------------------------------

\usepackage{epstopdf, longtable}
\usepackage{multicol}
\setlength{\columnsep}{3em}
\articletype{acb} % article type
% {inv} Investigation
% {gs} Genomic Selection
% {iba} Birkbeck, Infectious Bacteria and Antibiotics
% {gos} Genetics of Sex
% {mp} Multiparental Populations
\runningtitle{Advance Cell Biology - Exam Notes} % For use in the footer
\runningauthor{Student Id \textit{13703089}}

\title{\vspace*{2cm}\textcolor{black}{Advance Cell Biology - Exam Notes}\\
{\Large The Nucleus}}

\author[1]{Author: 13703089}
%\author[1]{Author Two}
%\author[2]{Author Three}
%\author[2,3]{Author Four}
%\author[4,$\ast$]{Author Five}

\affil[1]{BSc Biomedicine$^{\ast}$}
%\affil[2]{Author two affiliation}
%\affil[3]{Author three affiliation}
%\affil[4]{Author four affiliation}
%\affil[$\dagger$]{These authors contributed equally to this work.}

% Use the \equalcontrib command to mark authors with equal
% contributions, using the relevant superscript numbers
%\equalcontrib{1}
%\equalcontrib{2}

\correspondingauthoraffiliation[$\ast$]{Correspondence Address: Birkbeck University of London, Department of Biological Sciences, 14 Malet St, London WC1E 7HX. Email: \href{mailto:biosciences@bbk.ac.uk}{biosciences@bbk.ac.uk}}

\begin{abstract}
  \textcolor{color3}{\revRanGTP}
%   \textcolor{color3}{\revHistones}
%  \textcolor{color3}{\revNucleosome}
%  \textcolor{color3}{\revPolymerase}

\end{abstract}

\dates{\rec{xx xx, xxxx} \acc{xx xx, xxxx}}
\keywords{RanGTP; Keyword2; Keyword3}
%\keywords{Histones; Keyword2; Keyword3}
%\keywords{RNA Polymerase II; Keyword2; Keyword3}
%\keywords{Nucleosome; Keyword2; Keyword3}
%\keywords{RNA Polymerase II; Keyword2; Keyword3}



\begin{document}

% Use \inputfile{table,figures,...} in document to import tables and figures ..
\inputfile{revisionQs}
\inputfile{01transcription}
\inputfile{headings2}
\inputfile{headings3}
\inputfile{headings4}
\inputfile{headings5}
\inputfile{results}
\maketitle
\thispagestyle{firststyle}
%\slugnote
%\firstpagefootnote
\vspace{-13pt}% Only used for adjusting extra space in the left column of the first page

% Include  here Content 
%\topicTranscription
%\topicRegulation
%
\twocolstart
\section{Pictures}
The results and discussion should not be repetitive and give a factual presentation of the data with all tables and figures referenced. The discussion should not summarize the results but provide an interpretation of the results, and should clearly delineate between the findings of the particular study and the possible impact of those findings in a larger context. Authors are encouraged to cite recent work relevant to their interpretations.\\[1em]
\twocolend
%\figure1{0.34}{Caption}\\[1em]


Present and discuss results only once, not in both the Results and Discussion sections. It is acceptable to combine results and discussion in order  to be succinct.\\[4em]

%\vspace*{1em}
%\table1{\textwidth}{Student Grades per rank}


\bibliography{bib/acb_transcription}
%\bibliography{bib/acb_nucleus}
%\bibliography{bib/acb_nucleosomes}


\end{document} 