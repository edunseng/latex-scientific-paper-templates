\def\materials {
\subsection{Cryo-preservstion of Mycobacterial cultures}
A culture of actively replicating \textit{M. Smegmatis} was preserved in 25\% (v/v) glycerol and cryo-preserved in the freezer at -80°C, for about 6-12 months. 
\subsection{Starter culture }
1\% (v/v) of the cryo-preserved M.Smegmatis mc2155 culture was added to the Middlebrook liquid medium (7H9) and enriched with Albumine Dextrose Catalyst (ADC). It was then grown until an Optical Density (OD) reading of OD600 = 1. One (1) mL was then transferred to a fresh 7H9 Middlebrook broth containing the ADC supplement and the process was repeated at least three times. Followed by inoculating a 1\% (v/v) of the culture into 100 mL 7H9 Middlebrook broth containing 10\% (v/v) of the ADC (Difco) supplement and 0.05\% (v/v) Tween 80 (Sigma). The starting culture was then left to incubate in a 50 mL falcon tube at 37°C and constantly shaken at 180 rpm in a shaker to avoid floating biofilm formation.

\subsection{Treatment of the Culture with INH, ETH and NCE}
Optical density (OD) readings were taken using an Eppendorf Bio Photometer at 600 nm. At an OD600 of 0.4, the starting culture was allocated into 4 separate falcon tubes and treated with 250 µg/mL INH, 500µg/mL ETH and 500µg/mL NCE and then returned to the shaker for 6 hours.

\subsection{Determination of Growth}
A four (4) – fold dilution of the culture was performed prior to the readings. After zeroing against the liquid broth media only, individual readings for the three treated samples with INH, ETH and NCE and the untreated wildtype (WT) were taken for $\lambda$ = 595 nm. Thereby loading a 1 mL sample from every culture into the Eppendorf Bio Photometer using 12.5 mm$^2$ cuvettes and paying attention to mix the culture thoroughly using a pipette to avoid false readings due to the bacillus tending to sediment towards the bottom of the culture.

\subsection{Preparation for ZN-staining}
Four (4) slides were labeled U (untreated), X (unknown), H (isoniazid) and E (ethionamide) and 50 µL of the respective bacterial culture was pipetted onto the slides to create a thin smear and allowed to fully air dry. Each smear was then inserted into the Genlab oven and heat-fixed at 80°C for 20 minutes.

\subsection{Ziel-Neelsen cold Staining from Culture}
The heat-fixed smears were then individually stained by covering every smear in carbol-fuchsin for 5 minutes using the TB-COLOUR Cold Staining Kit from the Bund Deutscher Hebammen Laboratory (BDH). Tap water was then used to carefully rinse the slides  for about 5 seconds until all excess dye is removed. The slides were then treated with a few drops of TB-COLOUR de-staining solution to completely remove all excess carbol-fuchsin dye. This step is then followed by another careful washing step using tap water. The stained slides are then counterstained with malachite green and leftto stain for aout 5 seconds before carefully rinsing again with tap water for another 5 seconds. The washed counterstained slides are then left to fully air dry.

\subsection{Mounting slides for microscopic observations}
To prevent contamination with the immersion oil of the 100x lens, one drop of Eukitt medium containing 45\% acrylic resin and 55\% xylenes is placed in the middle of the slide. The specimen is then covered with a cover slip using a needle to avoid bubble formation. The prepared slides are then left to solidify for 5 minutes.

\subsection{Examination under the microscope}
The slides are the individually placed under the bright-field microscope and observations made for x20, x40 and x100 (immersion oil) lenses, x10 ocular lens and the magnification caused by the refractive index of glass and Eukitt medium (~ 1.510) resulting in a maximal magnification of ~ x1000. All slides were observed using the immersion oil lens (x100) and recorded by a photograph. 

}
\endinput