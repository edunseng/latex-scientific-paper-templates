\documentclass[11pt]{cls/labreport}
%------------------------------------------------------------------------------------------------
% Use the documentclass option 'lineno' to view line numbers
%  COMMANDS:
% \articletype{styles} ==>>  will load  "geos, gs, iba, inv, mp" style files
% \inputfile{table,figures,...} in document ==> to load tables and figures from helper files ..
% \fullref{tab:table1 or fig:figure1} etc   ==>  to reference tables, figures, etc
% \cite(alt(p))[p.32]{Tewabe2021} ==>  (Jones et al., 1990, P32)
% \table1{\textwidth}{Caption}              ==>  insert tables
% \figure{0.34}{Caption}                    ==>  insert figures
% \twocolstart                              ==>  switch to two column mode 
% \twocolstop                               ==>  switch to one column mode 
%-------------------------------------------------------------------------------------------------

\usepackage{epstopdf}
\usepackage{multicol}
\usepackage[backend=biber,style=apa]{biblatex} % Use APA style
\addbibresource{bib/iba.bib}  % Add the correct bibliography file

\setlength{\columnsep}{1.4em}
\articletype{iba} % article type
% {inv} Investigation
% {gs} Genomic Selection
% {iba} Birkbeck, Infectious Bacteria and Antibiotics
% {gos} Genetics of Sex
% {mp} Multiparental Populations
\runningtitle{SCBS080H6 (IBA) - Coursework} % For use in the footer
\runningauthor{Student Id \textit{13703089}}

\title{\vspace*{2cm} Mycobacterium smegmatis: Evaluation of isoniazid, ethionamide and a novel chemical entity using a modified Ziehl-Neelsen staining}

\author[1]{Author: 13703089}
%\author[1]{Author Two}
%\author[2]{Author Three}
%\author[2,3]{Author Four}
%\author[4,$\ast$]{Author Five}

\affil[1]{BSc Biomedicine$^{\ast}$}
%\affil[2]{Author two affiliation}
%\affil[3]{Author three affiliation}
%\affil[4]{Author four affiliation}
%\affil[$\dagger$]{These authors contributed equally to this work.}

% Use the \equalcontrib command to mark authors with equal
% contributions, using the relevant superscript numbers
%\equalcontrib{1}
%\equalcontrib{2}

\correspondingauthoraffiliation[$\ast$]{Correspondence Address: Birkbeck University of London, Department of Biological Sciences, 14 Malet St, London WC1E 7HX. Email: \href{mailto:biosciences@bbk.ac.uk}{biosciences@bbk.ac.uk}}
\keywords{Cold ZN-Staining; Acid-Fast Bacteria; Cell wall inhibitor; Mycolic acid synthesis }

%\begin{abstract}
%    \abstract
%\end{abstract}


\dates{\rec{xx xx, xxxx} \acc{xx xx, xxxx}}


\begin{document}

% Use \inputfile{table,figures,...} in document to import tables and figures ..
\inputfile{abstract}
\inputfile{introduction}
\inputfile{materials}
\inputfile{results}
\inputfile{definitions}
\inputfile{riskassesment}
\maketitle
\thispagestyle{firststyle}
%\slugnote
%\firstpagefootnote
\vspace{-13pt}% Only used for adjusting extra space in the left column of the first page

\begin{multicols}{2}
\section{Introduction}
    \introduction
    
\section{Risk assesment and waste disposal}
    \riskassesment
    
\section{Materials and methods}
\label{sec:materials:methods}
    \materials

\section{Results}
    \results


\section{Discussion}
After the ZN-staining and decolorization procedure the untreated sample appeared pink (\fullref{fig:figure6}), while the treated samples appeared green (\fullref{fig:figure7}, \fullref{fig:figure8}). This suggests the formation of a mycolate-fuchsin complex in the cell wall of the \textit{M. smegmatis} wild type and a lack thereof in the groups treated with INH, ETH, and the NCE. Isoniazid and Ethionamide are known narrow spectrum, bactericidal, anti-TB drugs which affect the bacterial viability by inhibiting a correct mycolic acid synthesis in the cell wall \cite{Banerjee1994,York2008}. This leads to our hypothesis, that the treated samples could not retain the pink stain, because they lost the protective effect from the hydrophobic, tightly packed, and rigid mycolic acid outer layer leading to the cells being susceptible for treatment with acids or alcohols as contained in the decolorising solution. The morphology showed a reduction in clumping and number of bacilli, which could be verified by an OD reduction of 63.3\% on average for INH, ETH and the NCE.  This suggests a potential bactericidal mode of action of the NCE, because it showed similar morphology to INH and ETH and the difference in OD was not significant between the three treatment groups.

\section{Conclusion}
Cold ZN-Staining  could successfully identify \textit{M. smegmatis} as an acid-fast bacterium (ACB). The unknown drug (NCE) could significantly reduce the number of viable \textit{M. smegmatis} bacilli and is therefore a promising new candidate for the treatment of TB caused by \textit{M. tuberculosis}. 
We can conclude this from our findings because\textit{ M. smegmatis} could be successfully decolorised after treatment with all three drugs. This result can be attributed to the inhibition of the cell wall formation by  the treated group. The morphology showed a reduction in clumping for all drugs. Mycolic acids stitch together and form clumps during normal growth, as seen in the untreated sample. The reduction in the bacilli numbers, which was verified by an average significant OD reduction of 63.3\%. The antibiotic treatment rank between the groups showed no significance, indicating that all three drugs have a similar mode of action. Considering  INH and ETH being bactericidal, we assume a bactericidal mode of action of the NCE.

\section{Acknowledgments}
The author acknowledges the PhD students and laboratory staff involved in organizing the laboratory session for cryo-preservation of\textit{ M. smegmatis} cultures, the preparation of the starter cultures and the treatment of the culture with INH, ETH, and Unknown Drug (NCE). The author also acknowledges Dr. Sanjib Bhakta, and the laboratory groups for supplying the figures and OD data used in this article. 

% Create the bibliography
\printbibliography

\end{multicols}
\end{document} 